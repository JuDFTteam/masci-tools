%% It is just an empty TeX file.
%% Write your code here.

\section{Theory}


\begin{frame}\frametitle{What is \textbf{in} the data?}
\begin{itemize}

\item Fleur solves the Schrödinger equation for electrons in crystals
\item \(\rightarrow\) approximate solutions applying Density Functional Theory (Kohn – Sham equation)

\item Output includes:
\item Bandstructure $E_n(k)$:
\begin{itemize}
    \item Eigenenergies of (Bloch-) Eigenfunctions of the Hamiltonian for each (crystal-) momentum $k$
    \item Disperion relation: Encodes, which energies (Bloch-) electrons in the solid are allowed to have
    \item e.g. allows distinction between conductor and semiconductor
    \item Sampled along a 1d Path between high symmetry points in 3d reciprocal space 
\end{itemize}

\item (Picture fcc real/reciprocal: idea: Schrödinger eqn solved in reciprocal space --> because of translation symmetry...)

\item Bandstructure D(E):
\begin{itemize}
    \item encodes, how many states are allowed within an Energy intervall
\end{itemize}

\end{frame}
\begin{frame}
    \begin{itemize}


\item Interesting for Physicist: Where do the contributions to E(k) and D(E) come from?

\item Data contains projection of the Eigenfunctions on Basis functions of the DFT calculation belonging to different atomgroups and atomic orbitals (s, p, d, f) in the unit cell
\item User might be interested in any superposition of them (e.g. to locate states in real space)
\item Information stored in form of weights for all Atom Groups and localized atomic orbitals s, p, d, f
    \end{itemize}

\end{itemize}
\end{frame}